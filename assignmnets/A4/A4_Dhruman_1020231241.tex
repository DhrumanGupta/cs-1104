\documentclass[a4paper]{article}
\setlength{\topmargin}{-1.0in}
\setlength{\oddsidemargin}{-0.2in}
\setlength{\evensidemargin}{0in}
\setlength{\textheight}{10.5in}
\setlength{\textwidth}{6.5in}
\usepackage{enumitem}
\usepackage{amsmath}
\usepackage{hyperref}
\usepackage{amssymb}
\usepackage{mathtools}
\usepackage{minted}
\usepackage[dvipsnames]{xcolor}
\usepackage{mathpartir}
\newlist{sollist}{itemize}{1}
\setlist[sollist]{label=$\implies$}
\usepackage{tikz}
\usetikzlibrary{positioning}

\makeatletter
\renewcommand*\env@matrix[1][*\c@MaxMatrixCols c]{%
  \hskip -\arraycolsep
  \let\@ifnextchar\new@ifnextchar
  \array{#1}}
\makeatother

\hypersetup{
    colorlinks=true,
    linkcolor=blue,
    filecolor=magenta,      
    urlcolor=cyan,
    pdftitle={Assignment 4},
    pdfpagemode=FullScreen,
    }
\def\endproofmark{$\Box$}
\newenvironment{proof}{\par{\bf Proof}:}{\endproofmark\smallskip}
\begin{document}
\begin{center}
{\large \bf \color{red}  Department of Computer Science} \\
{\large \bf \color{red}  Ashoka University} \\

\vspace{0.1in}

{\large \bf \color{blue}  Discrete Mathematics: CS-1104-1 \& CS-1104-2}

\vspace{0.05in}

    { \bf \color{YellowOrange} Assignment 4}
\end{center}
\medskip

{\textbf{Collaborators:} None} \hfill {\textbf{Name: Dhruman Gupta} }

\bigskip
\hrule


% Begin your assignment here %


\section{Straightforward}
\begin{enumerate}
    \item (a) The Master Theorem provides solutions to recurrences of the form $T(n) = aT(n/b) + f(n)$, given that $T(1) = c$. $f(n) \in \Theta (n^d)$. From the equation $T(n) = 4T(n/2) + c$, we get that:\\
    $$a = 4,\ b = 2,\ f(n) = c, d = 1$$
    These satisfy the conditions $a \geq 1,\ b > 1,\ c > 0$. \\
    As $a > b^d \leftrightarrow 4 > 2^1$, we are in case 1 of the Master Theorem.\\
    So, $T(n) \in \Theta(n^{\log_2 4}) = \Theta(n^2)$.\\

    (b) $T(n) = T(n/4) + T(n/2) + nc$, where $T(1) = c_0$. \\
    In order to apply the Master Theorem, we need to convert the recurrence into the form $T(n) = aT(n/b) + f(n)$.\\

    So, we can take: $T(n) \leq 2T(n/2) + nc$. Applying the Master Theorem, we get that:
    $$a = 2,\ b = 2,\ f(n) = nc, d = 1$$

    These satisfy the conditions $a \geq 1,\ b > 1,\ c > 0$. \\

    As $a = b^d \leftrightarrow 2 = 2^1$, we are in case 2 of the Master Theorem.\\

    So, $T(n) \in \Theta(n^d \log_2 n) = \Theta(n \log_2 n)$.\\


    (c) $T(n) = T(n-2) + T(n-4)$, where $T(0) = T(1) = T(3) = c_0$.\\
    We can convert this into the form $T(n) = aT(n/b) + f(n)$ by taking $T(n) = T(n-2) + T(n-4) \leq 2T(n-2)$.\\

    So, we get $a = 2,\ b = 1,\ f(n) = 0, d = 0$. This does not satisfy the conditions of the Master Theorem. So, manually evaluating the recurrence, we get:
    \begin{sollist}
    \item $T(n) \leq 2T(n-2)$ 
    \item $\leq 4T(n-4)$
    \item $\leq 8T(n-6)$
    \item $\leq \dots$
    \item $\leq 2^kT(n-2k)$ (after $k$ steps)\\
\end{sollist}
We know there will thus be $\frac{n}{2}$ steps. So, $T(n) = 2^{\frac{n}{2}}T(0) = 2^{\frac{n}{2}}c_0$.\\
So, $T(n) \in \Theta(2^n)$.\\

    \item (a)
    \begin{enumerate}[label=\roman*]
        \item \textbf{Definition}:\\
- Input: Integer $m, n \geq 0$\\
- Output: Integer $x$ such that $x = m \times n$.
\item \textbf{Decomposition}: We know that $m \times n = m \times (n-1) + m$. So, we can write our problem as:\\
\begin{verbatim}
def multiply(m, n):
    if n == 0:
        return 0
    return multiply(m, n-1) + m
\end{verbatim}
\item \textbf{Deconstruction}: From our decomposition, we know:
\begin{center}
    $T(n) = T(n-1) + O(p)$\\
$T(0) = O(1)$
\end{center}
\begin{sollist}
    \item $T(n-1) + O(p)$
    \item $T(n-2) + 2O(p)$
    \item $T\dots$
    \item $T(0) + nO(p)$
    \item $nO(p)$
\end{sollist}
Now, we know that $p = \log_2 n$, so we get $T(n) = nO(\log_2 n)$.\\
Thus, $T(n) \in \Theta (n \log_2 n)$. Alternatively, we get $T(p) \in \Theta(p * 2^p)$, where $p$ is the number of bits.
\item \textbf{Design}: We cannot use memoization to improve the time complexity of this algorithm.
\item \textbf{Iteration}: We can write the function iteratively as follows:
\begin{verbatim}
    def multiply(m, n):
        result = 0
        while n > 0:
            result += m
            n -= 1

        return result
\end{verbatim}
\item \textbf{Correctness}:
$$P(n): multiply(m, n) = m\times n$$
\textbf{Base Case}: $n = 0$. The function returns 0, which is the correct answer, as $m\times 0 = 0,\ \forall m$. \\
\textbf{Inductive Hypothesis}: Assume the function works for some $k$, i.e $P(k)$ holds.\\
\textbf{Inductive Step}: Need to show that $P(k+1)$ holds.\\
By definition, we have:
\begin{center}
    
$multiply(m, k+1) = multiply(m, k) + m$\\
$= m \times k + m\ \ \text{ (IH)}$ \\
$= m \times (k+1)$
\end{center}
Thus, $P(k+1)$ holds. Hence, $P(k) \rightarrow P(k+1)$.
    \end{enumerate}

(b)
\begin{enumerate}[label=\roman*]
    \item \textbf{Definition}:\\
    - Input: Integer $m > 0,\ n \geq 0$\\
    - Output: Integer $x$ such that $x = m^n$.
    \item \textbf{Decomposition}: We know that $m^n = m^{n-1} \times m$. So, we can write our problem as:\\
    \begin{verbatim}
    def power(m, n):
    if n == 0:
        return 1
    return power(m, n-1) * m
\end{verbatim}

\item \textbf{Deconstruction}: From our decomposition, we know:
\begin{center}
    $T(n) = T(n-1) + O(p^2)$\\
$T(0) = O(1)$
\end{center}

\begin{sollist}
    \item $T(n-1) + O(p^2)$
    \item $T(n-2) + 2O(p^2)$
    \item $T\dots$
    \item $T(0) + nO(p^2)$
    \item $nO(p^2)$
\end{sollist}

Now, we know that $p = \log_2 n$, so we get $T(n) = nO((\log_2 n)^2)$.\\
Thus, $T(n) \in \Theta (n (\log_2 n)^2)$. Alternatively, we get $T(p) \in \Theta(p^2 \times 2^p)$, where $p$ is the number of bits.

\item \textbf{Design}: We cannot use memoization to improve the time complexity of this algorithm.
\item \textbf{Iteration}: We can write the function iteratively as follows:
\begin{verbatim}
    def power(m, n):
        result = 1
        while n > 0:
            result *= m
            n -= 1

        return result
\end{verbatim}

\item \textbf{Correctness}:
$$P(n): power(m, n) = m^n$$
\textbf{Base Case}: $n = 0$. The function returns 1, which is the correct answer, as $m^0 = 1,\ \forall m$. \\
\textbf{Inductive Hypothesis}: Assume the function works for some $k$, i.e $P(k)$ holds.\\
\textbf{Inductive Step}: Need to show that $P(k+1)$ holds.\\
By definition, we have:
\begin{center}
    $power(m, k+1) = power(m, k) \times m$\\
    $= m^k \times m\ \ \text{ (IH)}$ \\
    $= m^{k+1}$
\end{center}
Thus, $P(k+1)$ holds. Hence, $P(k) \rightarrow P(k+1)$.\\
\end{enumerate}
(c)
\begin{enumerate}[label=\roman*]
    \item \textbf{Definition}:\\
    - Input: String s with first out of bound character as the literal '$\backslash 0$' \\
    - Output: Integer $x$ where $x$ is the length of the string $s$.

    \item \textbf{Decomposition}: We know that the length of a string is the length of the string without the first character + 1. So, we can write our problem as:
\begin{verbatim}
def length(s):
    def helper(s, i):
        if s[i] == '\0':
            return 0
        return 1 + helper(s, i+1)
    return helper(s, 0)
\end{verbatim}

\item \textbf{Deconstruction}: From our decomposition, we know:
\begin{center}
    $T(n) = T(n-1) + O(1) +O(1)$\\
    $T(0) = O(1)$
\end{center}
Where $n$ is the size of the string $s$.
\begin{sollist}
    \item $T(n-1) + 2O(1)$
    \item $T(n-2) + 4O(1)$
    \item $T\dots$
    \item $T(0) + nO(1)$
    \item $nO(1)$
\end{sollist}

Thus, $T(n) \in \Theta(n)$.\\

\item \textbf{Design}: We cannot use memoization to improve the time complexity of this algorithm.

\item \textbf{Iteration}: We can write the function iteratively as follows:
\begin{verbatim}
    def length(s):
        i = 0
        while s[i] != '\0':
            i += 1

        return i
\end{verbatim}

\item \textbf{Correctness}:
$$P(n): length(s) = n$$
\textbf{Base Case}: $n = 0$. So, $s = "\backslash 0"$. Thus, $s[0] = '\backslash 0'$. The function returns 0. \\
\textbf{Inductive Hypothesis}: Assume the function works for some $k$, i.e $P(k)$ holds.\\
\textbf{Inductive Step}: Need to show that $P(k+1)$ holds.\\
By definition, we have:
\begin{sollist}
    \item $length(s) = 1 + length(s[1:])$
    \item $1 + k\ \ \text{ (IH)}$
    \item $k+1$
\end{sollist}

Thus, $P(k+1)$ holds. Hence, $P(k) \rightarrow P(k+1)$.\\

\end{enumerate}

(d)
\begin{enumerate}[label=\roman*]
    \item \textbf{Definition}:\\
    - Input: Integer $a \geq 0,\ b > 0$\\
    - Output: Integer $x$ where $x = a \mod b$

    \item \textbf{Decomposition}: We know that $a \mod b = (a-b) \mod b$. So, we can write our problem as:
\begin{verbatim}
def mod(a, b):
    if a < b:
        return a
    return mod(a-b, b)
\end{verbatim}

\item \textbf{Deconstruction}: From our decomposition, we know:
\begin{center}
    $T(a) = T(a-b) + O(p) + O(1)$\\
    $T(0) = O(1)$
\end{center}
Where $p$ is the max of the bits in $a$ and $b$. Solving the recurence:
\begin{sollist}
    \item $\leq T(a-b) + 2O(p)$
    \item $T(a-2b) + 4O(p)$
    \item $T\dots$
    \item $T(0) + 2 \frac{a}{b}O(p)$
    \item $2 \frac{a}{b}O(p)$
\end{sollist}
We also know that $p = \log_2 a$.
Thus, $T(a) \in \Theta(\frac{a}{b} p) = \Theta(a \log_2 a)$.\\

\item \textbf{Design}: We cannot use memoization to improve the time complexity of this algorithm.

\item \textbf{Iteration}: We can write the function iteratively as follows:
\begin{verbatim}
    def mod(a, b):
        while a >= b:
            a -= b

        return a
\end{verbatim}

\item \textbf{Correctness}: We are iterating on $a$, as $b$ is invariant throughout.
$$P(a, b): mod(a, b) = a \mod b$$
\textbf{Base Case}: $a < b$. The function returns $a$, which is the correct answer, as $a \mod b = a,\ \forall a < b$. \\
\textbf{Inductive Hypothesis}: Assume by strong induction, that the function holds $\forall i, 0 \leq 0 \leq k$, for some $k$, i.e $P(k, b)$ holds.\\
\textbf{Inductive Step}: Need to show that $P(k+1, b)$ holds.\\
By definition, we have:
\begin{sollist}
    \item $mod(k+1, b) = mod(k+1-b, b)$
    \item $(k+1-b) \mod b\ \ \text{ (IH)}$
    \item $(k+1) \mod b$
\end{sollist}

Thus, $P(k+1, b)$ holds. Hence, $P(k, b) \rightarrow P(k+1, b)$.\\
\end{enumerate}
\vspace{1.5in}

(e)
\begin{enumerate}[label=\roman*]
    \item \textbf{Definition}:\\
    - Input: Integer Array $A$, Integer $n$, denoting size of $A$\\
    - Output: Integer Array $A$ such that $A$ is sorted in ascending order.

    \item \textbf{Decomposition}: We know that the smallest element in an array is the smallest element in the array without the first element. So, if we can assert that the first $i$ elements of the array are sorted, the sorting problem can be narrowed down to a size of $n-i$. So, we can write our problem as:

\begin{verbatim}
def sort(A, n, i=0):
    def subsort(A, j):
        if j > 0 and A[j-1] > A[j]:
            A[j-1], A[j] = A[j], A[j-1]
            return subsort(A, j-1)
        return A
    
    if i == n:
        return A
    
    return sort(subsort(A, i), n, i+1)
\end{verbatim}

\item \textbf{Deconstruction}: From our decomposition, we first need to solve the time complexity of the subsort function.\\
\begin{center}
    $T(n) = T(n-1) + O(1)$\\
    $T(0) = O(1)$
\end{center}
\begin{sollist}
    \item $T(n-1) + O(1)$
    \item $T(n-2) + 2O(1)$
    \item $T\dots$
    \item $T(0) + nO(1)$
    \item $nO(1) = O(n)$
\end{sollist}

Now, we can solve the time complexity of the sort function.\\
\begin{center}
    $T(n) = T(n-1) + O(n) + O(1)$\\
    $T(0) = O(1)$
\end{center}

\begin{sollist}
    \item $T(n) \leq T(n-1) + 2O(n)$
    \item $T(n-2) + 4O(n)$
    \item $\dots$
    \item $T(0) + nO(n)$
    \item $nO(n) = O(n^2)$
\end{sollist}

Thus, $T(n) \in \Theta(n^2)$.\\

\item \textbf{Design}: We cannot use memoization to improve the time complexity of this algorithm.

\item \textbf{Iteration}: We can write the function iteratively as follows:

\begin{verbatim}
def sort(A, n):
    i = 0
    while i < n:
        j = i
        while j > 0 and A[j-1] > A[j]:
            A[j-1], A[j] = A[j], A[j-1]
            j -= 1
        i += 1
    return A
\end{verbatim}

\item \textbf{Correctness}:
$$P(n): sort(A, n) = A_{\text{sorted}} \text{, where } A_{\text{sorted}} \text{ is A sorted}$$

\textbf{Base Case}: $n = 0$. The function returns $A$, which is the correct answer, as an empty array is sorted. \\
\textbf{Inductive Hypothesis}: Assume the function works for some $k$, i.e $P(k)$ holds.\\
\textbf{Inductive Step}: We need to prove 2 algorithms here. The inner loop, and the outer loop.\\

\textbf{Inner Loop}:\\
$$P(j): \text{The first j elements of A are sorted}$$
\textbf{Base Case}: $j = 0$. The function returns $A$, which is the correct answer, as the first element is a singleton, and thus sorted. \\
\textbf{Inductive Hypothesis}: Assume the function works for some $k$, i.e $P(k)$ holds.\\
\textbf{Inductive Step}: Need to show that $P(k+1)$ holds.\\

Case 1, $A[j+1] \geq A[j]$. This means $P(j+1)$ is already true.\\

Case 2, $A[j+1] < A[j]$. So, we need to show that $A[j+1]$ is moved to the correct position. This is done by the swapping operation. So, $P(j+1)$ holds.\\

Thus, $P(j) \rightarrow P(j+1)$, and so, the inner loop is correct.\\

\textbf{Outer Loop}:\\
$$P(n): sort(A, n) = A_{\text{sorted}}$$

\textbf{Base Case}: $n = 0$. The function returns $A$, which is the correct answer, as an empty array is sorted. \\

\textbf{Inductive Hypothesis}: Assume the function works for some $k$, i.e $P(k)$ holds.\\

\textbf{Inductive Step}: Need to show that $P(k+1)$ holds. By definition, we have:\\

$sort(A, k+1) = sort(A_1, k+1)$, where $A_1$ is the array after the first iteration of the inner loop.

We know that the singleton sub-array consisting of the element $A[0]$ is sorted (as our inner loop is correct). So, the problem becomes $sort(A_1, k)$. By the inductive hypothesis, we know that $sort(A_1, k) = A_{\text{sorted}}$. So, $sort(A, k+1) = A_{\text{sorted}}$.\\

Hence, $P(k) \rightarrow P(k+1)$.\\
\end{enumerate}
\end{enumerate}

\section{$\neg$ Straightforward}
\begin{enumerate}
    \item Need to solve the towers of Hanoi problem.
\begin{enumerate}[label=\roman*]
    \item \textbf{Definition}:\\
    - Input: Integer $n$, denoting the size of the problem \\
    - Output: Moves to move the $n$ disks from A to C, s.t. no larger disk is placed on a smaller disk.

    \item \textbf{Decomposition}: We know that the $n$th disk can be moved to C only if the first $n-1$ disks are moved to B. We move the first $n-1$ disks to the B. Then, we move the last disk to C. Now, we have no disks in tower A, $n-1$ in B, and the heaviest in $C$. So, if we say that A is our auxilary tower, we already know how to solve for this $n-1$ case. So, we can write our problem as:
\begin{verbatim}
def hanoi(n, start="A", aux="B", end="C"):
    if n == 1:
        print("Move disk 1 from ", start, " to tower ", end)
        return
    hanoi(n-1, start, end, aux)
    print("Move disk", n, "from", start, "to", end)
    hanoi(n-1, aux, start, end)
\end{verbatim} 

\item \textbf{Deconstruction}: From our decomposition, we know:
\begin{center}
    $T(n) = 2T(n-1) + O(1)$\\
    $T(1) = O(1)$
\end{center}

\begin{sollist}
    \item $T(n) = 2T(n-1) + O(1)$
    \item $2(2T(n-2) + O(1)) + O(1)$
    \item $4T(n-2) + 3O(1)$
    \item $8T(n-2) + 7O(1)$
    \item $\dots$
    \item $2^kT(n-k) + (2^k - 1)O(1)$
\end{sollist}

So when $k = n-1$, we get:\\

$T(n) = 2^{n-1}T(1) + (2^{n-1} - 1)O(1)$\\
$= 2^{n-1}O(1) + (2^{n-1} - 1)O(1)$\\
$= O(2^n)$
$= \Theta(2^n)$

\item \textbf{Design}: We cannot directly memoize the state of this algorithm, as the same state is not computed again.
\item \textbf{Iteration}: We can write the function iteratively as follows:

\begin{verbatim}
    def hanoi(n, start="A", aux="B", end="C"):
        if n == 1:
            print("Move disk 1 from ", start, " to tower ", end)
            return
        stack = []
        stack.append((n, start, aux, end))
        while stack:
            n, start, aux, end = stack.pop()
            if n == 1:
                print("Move disk 1 from ", start, " to tower ", end)
            else:
                stack.append((n-1, aux, start, end))
                stack.append((1, start, aux, end))
                stack.append((n-1, start, end, aux))
\end{verbatim}

This is a non-recursive implementation of the towers of Hanoi problem. It uses the stack data structure to simulate the recursive calls, and uses a depth-first search to solve the problem iteratively.

\item \textbf{Correctness}:
$$P(n): hanoi(n, s, a, e) = \text{Correct moves to solve the towers of Hanoi problem, given s, a, e = A, B, C respectively.}$$
\textbf{Base Case}: $n = 1$. The function prints the correct move, which is the correct answer, as the base case is trivial. \\
\textbf{Inductive Hypothesis}: Assume the function works for some $k$, i.e $P(k)$ holds.\\
\textbf{Inductive Step}: Need to show that $P(k+1)$ holds.\\
By definition, we have:
\begin{sollist}
    \item $hanoi(k+1, s, a, e) = hanoi(k, s, e, a) + \text{Move disk from A to C} \text{ to C} + hanoi(k, a, s, e)$
    \item $= \text{Correct moves to move first K disks from s to a} + \text{Correct move to move largest disk from s to e} + \text{Correst moves to move K disks from a to e}$ (by IH)
\end{sollist}

Logically, this moves the first $k$ disks to the auxilary tower, moves the largest disk to the end tower, and then moves the $k$ disks from the auxilary tower to the end tower. This is the correct solution to the towers of Hanoi problem.\\

Thus, $P(k) \rightarrow P(k+1)$.\\

\end{enumerate}

\item Need to implement merge sort.

\begin{enumerate}[label=\roman*]
    \item \textbf{Definition}:\\
    - Input: Integer Array $A$, Integer $n \geq 0$, denoting the size of $A$\\
    - Output: Integer Array $A$ such that $A$ is sorted in ascending order.

    \item \textbf{Decomposition}: Merge sort is a divide and conquer strategy, that works by dividing the array into two halves, sorting the two halves, and then merging the two halves. So, we can write our problem as:
    
\vspace{1in}
\begin{verbatim}
def merge_sort(A, n):
    def merge(left, right, acc=[]):
        if len(left) == 0:
            return acc + right

        if len(right) == 0:
            return acc + left

        if left[0] < right[0]:
            acc.append(left[0])
            return merge(left[1:], right, acc)
        else:
            acc.append(right[0])
            return merge(left, right[1:], acc)

    if n <= 1:
        return A
    mid = n // 2
    left = merge_sort(A[:mid], mid)
    right = merge_sort(A[mid:], n-mid)
    return merge(left, right)
\end{verbatim}

\item \textbf{Deconstruction}: From our decomposition, we know:
\begin{center}
    $T(n) = 2T(n/2) + O(n)$\\
    $T(1) = O(1)$
\end{center}

\begin{sollist}
    \item $T(n) = 2T(n/2) + O(n)$
    \item $2(2T(n/4) + O(n/2)) + O(n)$
    \item $4T(n/4) + 2O(n/2) + O(n)$
    \item $8T(n/8) + 4O(n/4) + 3O(n/2) + O(n)$
    \item $\dots$
    \item $2^kT(n/2^k) + kO(n/2^k)$
\end{sollist}

When $n = 2^k, i.e k = \log_2 n$, we get:\\
$2^{\log_2 n} T(n/2^{\log_2 n}) + \log_2 n O(n/2^{\log_2 n})$\\
$= nT(1) + \log_2 n O(1)$\\
$= O(n \log_2 n)$\\
$= \Theta(n \log_2 n)$

\item \textbf{Design}: We cannot directly memoize the state of this algorithm, as no same state is computed again.

\item \textbf{Iteration}: We can write the function iteratively as follows:
\begin{verbatim}
def merge_sort(A, n):
    def merge(l, r):
        result = []
        i, j = 0, 0
        while i < len(l) and j < len(r):
            if l[i] < r[j]:
                result.append(l[i])
                i += 1
            else:
                result.append(r[j])
                j += 1

        if i < len(l):
            result.extend(l[i:])

        if j < len(r):
            result.extend(r[j:])
        return result

    if len(A) <= 1:
        return A

    mid = n//2
    left = A[:mid]
    right = A[mid:]

    return merge(merge_sort(left, mid), merge_sort(right, n-mid))
\end{verbatim}

\item \textbf{Correctness}:

$$P(n): merge\_sort(A, n) = A_{\text{sorted}} \text{, where } A_{\text{sorted}} \text{ is A sorted}$$
\textbf{Base Case}: $n \leq 1$. The function returns $A$, which is the correct answer, as the base case is trivial. \\
\textbf{Inductive Hypothesis}: Assume the function works for some $k$, i.e $P(k)$ holds.\\
\textbf{Inductive Step}: Need to show that $P(k+1)$ holds.\\
By definition, we have:
\begin{sollist}
    \item $merge\_sort(A, k+1) = merge(merge\_sort(A[:k], k), merge\_sort(A[k:], 1))$
    \item $merge(A_{\text{sorted}}[:k], A_{\text{sorted}}[k:])$ (IH)
    \item $A_{\text{sorted}}$
\end{sollist}

For this, however, we need to prove the correctness of the merge function. Let $n = len(L) + len(R)$, where $L$ and $R$ are two sorted arrays.\\
$$P(n): merge(L, R) = A, \text{s.t. A is sorted}$$

\textbf{Base Case}: $n = 0$. The function returns an empty array, which is the correct answer, as an empty array is sorted. \\

\textbf{Inductive Hypothesis}: Assume the function works for some $k$, i.e $P(k)$ holds.\\

\textbf{Inductive Step}: Need to show that $P(k+1)$ holds.\\

WLOG, assume $\min(L[0], R[0]) = L[0]$. By definition, we have:
\begin{sollist}
    \item $merge(L, R) = [L[0]] + merge(L[1:], R)$
    \item $[L[0]] + B_{sorted}$ (IH)
    \item $A_{\text{sorted}}$ (as $L[0] \leq R[0] \implies L[0] \leq B_{sorted}[i],\ \forall i \in [1, ..., k]$)
\end{sollist}

Thus, $P(k) \rightarrow P(k+1)$.\\


\end{enumerate}
\end{enumerate}
\end{document}