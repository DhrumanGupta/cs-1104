\documentclass[a4paper]{article}
\setlength{\topmargin}{-1.0in}
\setlength{\oddsidemargin}{-0.2in}
\setlength{\evensidemargin}{0in}
\setlength{\textheight}{10.5in}
\setlength{\textwidth}{6.5in}
\usepackage{enumitem}
\usepackage{amsmath}
\usepackage{hyperref}
\usepackage{amssymb}
\usepackage{mathtools}
\usepackage{minted}
\usepackage[dvipsnames]{xcolor}
\usepackage{mathpartir}
\newlist{sollist}{itemize}{1}
\setlist[sollist]{label=$\implies$}
\usepackage{tkz-berge}
\usepackage{tikz}
\usetikzlibrary{positioning}
\usetikzlibrary{graphs,graphs.standard}
\usetikzlibrary{automata, positioning, arrows}

\makeatletter
\renewcommand*\env@matrix[1][*\c@MaxMatrixCols c]{%
  \hskip -\arraycolsep
  \let\@ifnextchar\new@ifnextchar
  \array{#1}}
\makeatother

\hypersetup{
    colorlinks=true,
    linkcolor=blue,
    filecolor=magenta,      
    urlcolor=cyan,
    pdftitle={Assignment 6},
    pdfpagemode=FullScreen,
    }
\def\endproofmark{$\Box$}
\newenvironment{proof}{\par{\bf Proof}:}{\endproofmark\smallskip}
\begin{document}
\begin{center}
{\large \bf \color{red}  Department of Computer Science} \\
{\large \bf \color{red}  Ashoka University} \\

\vspace{0.1in}

{\large \bf \color{blue}  Discrete Mathematics: CS-1104-1 \& CS-1104-2}

\vspace{0.05in}

    { \bf \color{YellowOrange} Assignment 6}
\end{center}
\medskip

{\textbf{Collaborators:} None} \hfill {\textbf{Name: Dhruman Gupta} }

\bigskip
\hrule


% Begin your assignment here %


\section{Straightforward}
\begin{enumerate}
\item Consider the following grammars:
\begin{itemize}
\item $G_1 =< \{a, b, c\}, \{S, A\}, S, S \rightarrow aS\ |\ bA, A \rightarrow cA\ |\ \epsilon >$
\item $G_2 =< \{a, b, c\}, \{S, A, B, C\}, S, S \rightarrow A, A \rightarrow aA\ |\ B, B \rightarrow bC, C \rightarrow cC\ |\ \epsilon >$
\end{itemize}

(a) Check if the following strings belong to L(G1) and/or L(G2). If a string belongs to a language,
show its derivation. If a string does not belong to a language, justify.
\\

$L(G_1)$:\\
i. $S \rightarrow aS \rightarrow abA \rightarrow abcA \rightarrow abc$. Thus $abc \in L(G1)$.\\
ii. $S \rightarrow aS \rightarrow aaS \rightarrow aabA \rightarrow aabcA \rightarrow aabccA \rightarrow aabcccA \rightarrow aabccc$. Thus $aabccc \in L(G1)$.\\
iii. $S \rightarrow aS \rightarrow aaS \rightarrow aabA$. Thus $aacc \notin L(G1)$.\\
iv. $S \rightarrow aS \rightarrow abA$. Thus $abbc \notin L(G1)$.\\

$L(G_2)$:\\
i. $S \rightarrow A \rightarrow aA \rightarrow aB \rightarrow abC \rightarrow abcC \rightarrow abc$. Thus $abc \in L(G2)$.\\
ii. $S \rightarrow A \rightarrow aA \rightarrow aaA \rightarrow aaB \rightarrow aabC \rightarrow aabcC \rightarrow aabccC \rightarrow aabcccC \rightarrow aabccc$. Thus $aabccc \in L(G2)$.\\
iii. $S \rightarrow A \rightarrow aA \rightarrow aaA \rightarrow aabB$. Thus $aacc \notin L(G2)$.\\
iv. $S \rightarrow A \rightarrow aA \rightarrow aB \rightarrow abC$. Thus $abbc \notin L(G2)$.\\

(b) Reason if $G_1$ and $G_2$ are regular grammars according to Chomsky's Hierarchy.\\

A grammar is considered regular if all of its productions are of the form:
\begin{itemize}
    \item $A\rightarrow \epsilon$
    \item $A\rightarrow a$
    \item $A\rightarrow aB$
\end{itemize}

Now, analysing $G_1$:
$S \rightarrow aS\ |\ bA$ (1)\\
$A \rightarrow cA\ |\ \epsilon$ (2)\\

This fits the definition of a regular grammar. Thus, $G_1$ is a regular grammar.\\

Now, analysing $G_2$:\\
$S \rightarrow A$ (1)\\
$A \rightarrow aA\ |\ B$ (2)\\
$B \rightarrow bC$ (3)\\
$C \rightarrow cC\ |\ \epsilon$ (4)\\

Rules (1) and (2) do not fit the definition of a regular grammar. Thus, $G_2$ is not a regular grammar.\\

(d) In part (c), we prove that $L(G_1) = L(G_2)$. In part (b), we also showed that $G_1$ is a regular grammar, while $G_2$ is not. However, since $L(G_1) = L(G_2)$, this means that $L(G_2)$ is regular as well. This holds because every regular grammar is a type 2 grammar, while every type 2 grammar is not necessarily regular. Thus, $L(G_2)$ is a regular language.\\

\newpage

\item (a) Proof by counter example:\\

Consider the string $ba$. Now, we see that if $G_r \equiv \{A\rightarrow aB, B\rightarrow aB\ |\ bB\ |\ \epsilon\}$, we cannot generate the string $ba$. Proof:\\
$A \rightarrow aB$. This means that the string must start with an $a$. Thus, $ab \notin L(G_r)$.\\

Now, consider $G_l \equiv \{A\rightarrow Ba,\ B\rightarrow Ba\ |\ Bb\ |\ \epsilon\}$. We can generate the string $ba$ using this grammar. Proof:\\
$A \rightarrow Ba \rightarrow Bba \rightarrow ba$. Thus, $ba \in L(G_l)$.\\


So, we see that $L(G_l) \neq L(G_r)$.\\

(b) The steps below convert a left linear grammar to a right linear grammar:
\begin{itemize}
\item If the start symbol, say $S$, is on the right of any production rule in $G_r$, then we add a new start symbol $S'$, and define the rule: $S' \rightarrow S$. If $S \rightarrow \epsilon$, add $S' \rightarrow \epsilon$.
\item For every production in $G_r$ of the form $A\rightarrow xB$, we replace it with $B\rightarrow Ax$.
\item If there is a production of the form $A\rightarrow x$, we keep it as it is.\\
\end{itemize}

(c)
\begin{itemize}
\item From our definition, we do not need a new start symbol, as the start symbol is only on the left.
\item We have $A\rightarrow aB$, which we convert to $B\rightarrow Aa$.
\item We have $B\rightarrow aB\ |\ bB\ |\ \epsilon$, which we convert to $B\rightarrow Ba\ |\ Bb\ |\ \epsilon$.\\
\end{itemize}

Thus, our right linear grammar is $G_r \equiv \{B\rightarrow Ba\ |\ Bb\ |\ \epsilon,\ A\rightarrow Ba\}$.\\

(d) In part (b), we are essentially reversing the direction of the productions. Thus, if we reverse the direction of the productions in $G_r$, we get $G_l$. Now, we have to show that this is correct.\\

WLOG, take some sentence $w$ in $L(G_r)$. We can derive $w$ in $G_r$ by starting with the start symbol, and applying the productions in $G_r$ in the order they are given. Now, as per our rules, there is an exact corresponding set of reversed productions in $G_l$ where it is possible to reach back to S (via the new productions).
\newpage

\item (a) We can represent the DFA as below:
\begin{itemize}
    \item $q_0$: Initial state, no input read yet.
    \item $q_{ae}$: First character 'a', even number of characters read.
    \item $q_{ao}$: First character 'a', odd number of characters read.
    \item $q_{be}$: First character 'b', even number of characters read.
    \item $q_{bo}$: First character 'b', odd number of characters read.
\end{itemize}

Transitions:
\begin{itemize}
    \item From $q_0$ on 'a': Move to $q_{ao}$.
    \item From $q_0$ on 'b': Move to $q_{bo}$.
    \item From $q_{ae}$ on 'a': Stay in $q_{ae}$.
    \item From $q_{ae}$ on 'b': Move to $q_{bo}$.
    \item From $q_{ao}$ on 'a': Move to $q_{ae}$.
    \item From $q_{ao}$ on 'b': Stay in $q_{ao}$.
    \item From $q_{be}$ on 'a': Move to $q_{ao}$.
    \item From $q_{be}$ on 'b': Stay in $q_{be}$.
    \item From $q_{bo}$ on 'a': Stay in $q_{bo}$.
    \item From $q_{bo}$ on 'b': Move to $q_{be}$.
\end{itemize}

Accepting States
\begin{itemize}
    \item $q_{ae}$: Accepts if $w_1 = w_n$ and $|w| \equiv 0 \pmod{2}$.
    \item $q_{bo}$: Accepts if $w_1 \neq w_n$ and $|w| \equiv 1 \pmod{2}$.
\end{itemize}

(b) $M_s$:\\
\begin{tikzpicture}[->, >=stealth, auto, node distance=3cm, thick, initial text=, initial where=left]
    \node[state, initial] (q0) {$q_0$};
    \node[state, accepting] (qaa) [above right of=q0] {$q_{aa}$};
    \node[state] (qa) [right of=qaa] {$q_a$};
    \node[state, accepting] (qbb) [below right of=q0] {$q_{bb}$};
    \node[state] (qb) [right of=qbb] {$q_b$};

    \path   (q0) edge node {a} (qaa)
            (q0) edge node {b} (qbb)
            (qaa) edge[loop above] node {a} (qaa)
            (qaa) edge[bend left] node {b} (qa)
            (qa) edge[bend left] node {a} (qaa)
            (qa) edge[loop right] node {b} (qa)
            (qbb) edge[loop below] node {b} (qbb)
            (qbb) edge[bend left] node {a} (qb)
            (qb) edge[bend left] node {b} (qbb)
            (qb) edge[loop right] node {a} (qb);
\end{tikzpicture}

$M_d$:\\
\begin{tikzpicture}[->, >=stealth, auto, node distance=3cm, thick, initial text=, initial where=left]
    \node[state, initial] (q0) {$q_0$};
    \node[state] (qa) [above right of=q0] {$q_a$};
    \node[state, accepting] (qab) [right of=qa] {$q_{ab}$};
    \node[state] (qb) [below right of=q0] {$q_b$};
    \node[state, accepting] (qba) [right of=qb] {$q_{ba}$};

    \path   (q0) edge node {a} (qa)
            (q0) edge node {b} (qb)
            (qa) edge[loop above] node {a} (qa)
            (qa) edge[bend left] node {b} (qab)
            (qb) edge[loop below] node {b} (qb)
            (qb) edge[bend left] node {a} (qba)
            (qab) edge[loop right] node {b} (qab)
            (qab) edge[bend left] node {a} (qa)
            (qba) edge[loop right] node {a} (qba)
            (qba) edge[bend left] node {b} (qb);
\end{tikzpicture}

$M_e$:\\
\begin{tikzpicture}[->, >=stealth, auto, node distance=3cm, thick, initial text=, initial where=left]
    \node[state, initial] (q0) {$q_0$};
    \node[state] (qo) [right of=q0] {$q_o$};
    \node[state, accepting] (qe) [right of=qo] {$q_{e}$};

    \path   (q0) edge node {a, b} (qo)
            (qo) edge[bend left] node {a, b} (qe)
            (qe) edge[bend left] node {a, b} (qo);
\end{tikzpicture}

$M_o$:\\
\begin{tikzpicture}[->, >=stealth, auto, node distance=3cm, thick, initial text=, initial where=left]
    \node[state, initial] (q0) {$q_0$};
    \node[state, accepting] (qo) [right of=q0] {$q_o$};
    \node[state] (qe) [right of=qo] {$q_{e}$};

    \path   (q0) edge node {a, b} (qo)
            (qo) edge[bend left] node {a, b} (qe)
            (qe) edge[bend left] node {a, b} (qo);
\end{tikzpicture}

(c) We see that $L$ needs a string that has even length and the same start and end character, or a string that has odd length and different start and end characters. Thus, we can represent $L$ as $L = (L_s \cap L_e) \cup (L_d \cap L_o)$.\\

(d) From (c) we have $  $. We can thus exploit the properties of NFAs to construct an NFA for $L$.\\\



\newpage

\item Solving for $L_I$:\\
(a)\\
\begin{tikzpicture}[shorten >=1pt, node distance=3cm, on grid, auto]
    \node[state, initial] (q0) {$q_0$};
    \node[state] (q1) [right=of q0] {$q_1$};
    \node[state, accepting] (qf) [below=of q0] {$q_f$};

    \path[->]
    (q0) edge[bend left, above] node {0,1} (q1)
    (q1) edge[bend left, below] node {0} (q0)
    (q0) edge node {0, 1} (qf);
\end{tikzpicture}

(b) Consider the regular grammar $G \equiv \{S\rightarrow 1A \ |\ 0A,\ A\rightarrow 0B\ |\ \epsilon,\ B\rightarrow 1A \ |\ 0A\ |\ \epsilon \}$\\

Take $w_a = 10101$, and $w_r = 10111$. We can see that $w_a \in L_I$, while $w_r \notin L_I$. Now, let's derive $w_a$:\\

$S \rightarrow 1A \rightarrow 10B \rightarrow 101A \rightarrow 1010B \rightarrow 10101A \rightarrow 10101$. Thus, $w_a \in L_I$.\\

Now, let's derive $w_r$:\\

$S \rightarrow 1A \rightarrow 10B \rightarrow 101A$. Thus, $w_r \notin L_I$.\\

(c) By definition, any means any sentence $w \in L$ is non-empty and has a 0 in every even position.\\

Now, when we look at $L(N)$, we see that it accepts 0 at all positions, but only 1 at odd positions. Thus, $L(N) = L$. Now, when we see $L(G)$, it places no constraints on odd positions and requires '0' at even positions Thus, $L(G) = L$. Thus, $L(N) = L(G) = L$.\\

Solving for $L_{II}$:\\

States:\\
$q_{00}, q_{10}, q_{11}, \cdots$, where each $q_{ij}$ represents a string prefix with exactly $i$ ones and $j$ zeros.\\

Transitions:\\
$0$ transitions from $q_{ij}$ to $q_{i(j+1)}$\\
$1$ transitions from $q_{ij}$ to $q_{(i+1)j}$\\

Accepting States:\\
$q_{ij}$ is an accepting state if $|i - j| \leq 1$.\\

(b) Consider the following production rules for the regular grammar $G_{II}$:\\

For each variable $A_{ij}$, we have the following rule:\\
$A_{ij} \rightarrow 0A_{i(j+1)} | 1A_{(i+1)j} |$\\
$A_{ij} \rightarrow \epsilon$ if $|i-j| \leq 1$\\

Now, take $w_a = 01010$ and $w_r = 00011$. Deriving $w_a$:\\
$S \rightarrow 0A_{01} \rightarrow 01A_{11} \rightarrow 010A_{12} \rightarrow 0101A_{22} \rightarrow 01010A_{23} \rightarrow 01010$. Thus, $w_a \in L_{II}$.\\

Now, deriving $w_r$:\\
$S \rightarrow 0A_{01} \rightarrow 00A_{02} \rightarrow 000A_{03} \rightarrow 0001A_{13} \rightarrow 00011A_{23}$. Thus, $w_r \notin L_{II}$.\\

(c)
\end{enumerate} 

\section{$\neg$ Straightforward}
\begin{enumerate}
\item A language $L$ is regular if and only if there exists a DFA $M$ such that $L = L(M)$. So, lets construct a DFA $M$ for the language $L$:\\

States:\\
Let any state $q_{id}$ be denoted where i is the current length of the sentence being processed, and d is the difference between the number of 1s and 0s in the sentence.\\

Transitions:\\
0: transition from $q_{id}$ to $q_{i+1,d-1}$\\
1: transition from $q_{id}$ to $q_{i+1,d+1}$\\

Start State:\\
$q_{0,0}$\\

Accepting States:\\
$q_{n,d}, \forall d > 0$, where $n$ is the length of the sentence.\\

This is a DFA that accepts the language $L_n$. Thus, $L_n$ is regular.\\

(b) (i) The string u will have a length of $n-k$. According to the definition of $L_n$, every prefix $w_i$ ($0 \leq i \leq n$) contains more 1s than 0s. Thus for u, the prefix will have more 1s than 0s. Thus, $u \in L_{n-k}$.\\

(ii) Assume that some $u$ which starts with 0 is in $L_n$ for some n. Now, we know by definition that every prefix of $u$ has more 1s than 0s. Thus, the prefix of $u$ will have more 1s than 0s. However, this is a contradiction at $i = 1$, thus $u \notin L_n$.\\


\end{enumerate} 


\end{document}