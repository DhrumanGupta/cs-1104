\documentclass[a4paper]{article}
\setlength{\topmargin}{-1.0in}
\setlength{\oddsidemargin}{-0.2in}
\setlength{\evensidemargin}{0in}
\setlength{\textheight}{10.5in}
\setlength{\textwidth}{6.5in}
\usepackage{enumitem}
\usepackage{amsmath}
\usepackage{hyperref}
\usepackage{amssymb}
\usepackage{mathtools}
\usepackage{minted}
\usepackage[dvipsnames]{xcolor}
\usepackage{mathpartir}
\newlist{sollist}{itemize}{1}
\setlist[sollist]{label=$\implies$}
\usepackage{tkz-berge}
\usepackage{tikz}
\usetikzlibrary{positioning}
\usetikzlibrary{graphs,graphs.standard}
\usetikzlibrary{automata, positioning, arrows}
\usepackage{longtable}

\makeatletter
\renewcommand*\env@matrix[1][*\c@MaxMatrixCols c]{%
  \hskip -\arraycolsep
  \let\@ifnextchar\new@ifnextchar
  \array{#1}}
\makeatother

\hypersetup{
    colorlinks=true,
    linkcolor=blue,
    filecolor=magenta,      
    urlcolor=cyan,
    pdftitle={Assignment 6},
    pdfpagemode=FullScreen,
    }
\def\endproofmark{$\Box$}
\newenvironment{proof}{\par{\bf Proof}:}{\endproofmark\smallskip}
\begin{document}
\begin{center}
{\large \bf \color{red}  Department of Computer Science} \\
{\large \bf \color{red}  Ashoka University} \\

\vspace{0.1in}

{\large \bf \color{blue}  Discrete Mathematics: CS-1104-1 \& CS-1104-2}

\vspace{0.05in}

    { \bf \color{YellowOrange} Assignment 6}
\end{center}
\medskip

{\textbf{Collaborators:} None} \hfill {\textbf{Name: Dhruman Gupta} }

\bigskip
\hrule


% Begin your assignment here %


\section{Straightforward}
\begin{enumerate}
\item (a) Entropy is given by: $H(x) = -\sum_{i=1}^{n} p(x_i) \log_2 p(x_i)$. We have $X = 
\{a : 0.80, b : 0.10, c : 0.05, d : 0.03, e : 0.017, f : 0.003\}$ 

\begin{sollist}
    \item $H(X) = -0.80 \log_2 0.80 - 0.10 \log_2 0.10 - 0.05 \log_2 0.05 - 0.03 \log_2 0.03 - 0.017 \log_2 0.017 - 0.003 \log_2 0.003$
    \item $H(X) = 1.0827$
\end{sollist}

(b) Design a Huffman Code $C: X \rightarrow {0, 1}^+$ for $X$\\
Following the huffman encoding algorithm, we get the following encoding for the given probabilities:

\begin{sollist}
    \item $a : 0.80 \rightarrow 1$
    \item $b : 0.10 \rightarrow 00$
    \item $c : 0.05 \rightarrow 010$
    \item $d : 0.03 \rightarrow 0111$
    \item $e : 0.017 \rightarrow 01101$
    \item $f : 0.003 \rightarrow 01100$
\end{sollist}

(c) The average length $L(C)$ of the code is given by: $L(C) = \sum_{i=1}^{n} p(x_i) \cdot l(x_i)$ where $l(x_i)$ is the length of the encoding of $x_i$ in $C$.\\

\begin{sollist}
    \item $L(C) = 0.80 \cdot 1 + 0.10 \cdot 2 + 0.05 \cdot 3 + 0.03 \cdot 4 + 0.017 \cdot 5 + 0.003 \cdot 5$
    \item $L(C) = 1.16$
\end{sollist}

The efficiency of the code is given by: $\eta = \frac{H(X)}{L(C)}$
\begin{sollist}
    \item $\eta = \frac{1.0827}{1.16}$
    \item $\eta = 0.9339$
\end{sollist}

(d) Encoding $baaaacbadaafaaae$ to X:
\begin{sollist}
    \item $baaaacbadaafaaae \rightarrow 0011110100010111110110011101101$ 
\end{sollist}

Also, since no code is a prefix of another code, the code is uniquely decodable.

(e) FSM:
\begin{sollist}
    \item 1: a
    \item 0: check next item
    \item 0: b
    \item 1: check next item
    \item 0: c
    \item 1: check next item
    \item 1: d
    \item 0: check next item
    \item 1: e
    \item 0: f
\end{sollist}

\item \begin{align*}
    P(i) & = 0.2, & P(s) & = 0.2 \\
    P(e) & = 0.15 \\
    P(t) & = 0.1, & P(o) & = 0.1 \\
    P(u) & = 0.1, & P(a) & = 0.1 \\
    \end{align*}

\item We will prove this by proving both ways.\\

Forward: Prefix-free implies instantaneous.\\
Suppose C is a prefix-free code. Consider a sequrence of codewords. Since C is prefix-free, no codeword is a prefix of another. When we decode the sequence, we can decode each codeword as soon as we see it. Hence, C is an instantaneous code.\\

Backward: Instantaneous implies Prefix-free.\\
Now, suppose C is an instantaneous code. Consider a sequence of codewords. Since C is instantaneous, we can decode each codeword as soon as we see it. Since we can decode each codeword as soon as we see it, no codeword is a prefix of another. Hence, C is a prefix-free code.\\

Thus, we have proved that a code is prefix-free if and only if it is instantaneous.

\item TODO
\item TODO
\newpage 

\item (a) We need $r$ such that $2^r \geq n + r + 1$. We have $n = 4$. Thus, we need $r$ such that $2^r \geq 5 + r$. This solves to $r = 3$.\\

(b) The placement would be: $p_1 p_2 d_1 p_3 d_2 d_3 d_4$.

(c) The hamming code we get is:\\
\begin{longtable}[c]{| c | c |}
Dataword & Codeword\\
\hline
0000 & 0000000\\
0001 & 1101001\\
0010 & 0101010\\
0011 & 1000011\\
0100 & 1001100\\
0101 & 0100101\\
0110 & 1100110\\
0111 & 0001111\\
1000 & 1110000\\
1001 & 0011001\\
1010 & 1011010\\
1011 & 0110011\\
1100 & 0111100\\
1101 & 1010101\\
1110 & 0010110\\
1111 & 1111111
\end{longtable}

(d) (i) $aaabaaaadaaaaaaaafaaaaabaaacaa = 11100111101111111111101100111110011101011$\\
(ii) $0001\ 1100\ 1111\ 0111\ 1111\ 1111\ 0110\ 0111\ 1100\ 1110\ 1011$\\
Now, when we hamming encode them, we get:\\
$1101001\ 1000011\ 1111111\ 0110011\ 1111111\ 0111100\ 0010110\ 1100110\ 1110000\ 1010101\ 0011001$\\

\end{enumerate} 


\end{document}